\documentclass[11pt,a4paper]{article}
\usepackage[top=30pt,bottom=30pt,left=40pt,right=40pt]{geometry}
\usepackage{xcolor,fontawesome}
\usepackage[T1]{fontenc}
\usepackage[bitstream-charter]{mathdesign}
\linespread{1.1}


\usepackage{titlesec}
\titleformat{\section}
	{\large\scshape}{\thesection}{1em}{}[{\titlerule[0.8pt]}]
\titlespacing{\section}
    {0em}{1em}{1em}

\setlength\parindent{0pt}
\pagenumbering{gobble}

\title{\vspace{-40pt}
	Ikhan Choi\\[5pt]
	\small \faEnvelopeSquare\quad dlrgks623[at]gmail.com
	\quad$\cdot$\quad \faHome\quad ikhanchoi.github.io
	\quad$\cdot$\quad \faGithub\quad github.com/ikhanchoi
	\vspace{-5em}}
\date{}

\usepackage{enumitem}
\setlist[itemize]{label=,itemindent=-2em,leftmargin=4em}


\begin{document}
\maketitle


\section*{Education}
\begin{itemize}
\item
	\textbf{University of Tokyo}
	\hfill{\small Apr 2025 -- Now}\\
	Ph.D. program in Mathematical Science
\item
	\textbf{University of Tokyo}
	\hfill{\small Apr 2023 -- Mar 2025}\\
	M.S. in Mathematical Science
\item
	\textbf{Pohang University of Science and Technology}
	\hfill{\small Mar 2016 -- Aug 2022}\\
	B.S. in Mathematics\\
	Early graduation: leave of absence for three years including military service
\item
	\textbf{Seoul Science High School}
	\hfill{\small Mar 2013 -- Feb 2016}\\
	Specialized high school for gifted students
\end{itemize}



\section*{Research Interests}
\hspace{2em}
Functional analytic features of bivariant K-theories and noncommutative topology from the viewpoint of\par
\hspace{2em}
homotopy theory, and their applications to representation theory and mathematical physics.



\section*{Publications}
\begin{itemize}
\item
	\textbf{I. Choi},
	\emph{A solution to Haagerup's problem and positionve Hahn-Banach separation theorems in operator algebras},\\
	preprint, arXiv: 2501.16832. (Master's thesis)\\
	\textcolor{gray}{Solved a 50 years old open problem posed by U. Haagerup.}
\item
	\textbf{I. Choi},
	\emph{Curved folding and planar cutting of simple closed curve on a conical origami},\\
	Kodai Math. J., 39-3 (2016) 579-595. (High school thesis)\\
	\textcolor{gray}{Suggested a problem, in terms of classical differential geometry, of determining whether a closed curve on a plane can be realized as the intersection of a piecewise smooth isometric immersion of the plane and another plane embedded in $\mathbb{R}^3$.}
\end{itemize}



\section*{Academic Experiences}
\begin{itemize}
\item
	Senior thesis, POSTECH
	\hfill{\small Spring 2022}\\
	Title: \emph{Three perspectives on Bochner's theorem: from Herglotz representation to Pontryagin duality}\\
	Advisor: \emph{Younghwan Son}\\
	\textcolor{gray}{Investigated Bochner's theorem from three different viewpoints; complex analysis, probability theory, and representation theory using GNS construction. As an application, proved the Pontryagin duality theorem.}
%	Abstract:

\item
	Undergraduate Research Program, POSTECH
	\hfill{\small Fall 2019}\\
	Title: \emph{Global Existence of the Vlasov-Poisson System}\\
	Advisor: \emph{Donghyun Lee}\\
	\textcolor{gray}{Proved the local existence of the Vlasov-Poisson system and reviewed Schaeffer's paper on the global existence.}
%	Abstract:
%	\textcolor{lightgray}{This report reviews Schaeffer's paper, based on the exposition in Glassey's book. We prove the local and global existence problem for a classical solution of the Cauchy problem for the Vlasov-Poisson system.}
\item IBS-CGP Mathematics Festival (Research Experience Program), POSTECH
	\hfill{\small Aug 2018}\\
	Topic: \emph{Variations on a Theme: On the Dispersion of Waves}\\
	Advisor: \emph{Sung-Jin Oh}
\end{itemize}




\section*{Teaching Experiences}
\begin{itemize}
\item
	Teaching Assistant: Geometry II
	\hfill{\small Fall 2024}
\item
	Teaching Assistant: Complex Analysis I
	\hfill{\small Fall 2023}
\item
	Student Mentoring: MATH 423 Introduction to Differential Geometry
	\hfill{\small Fall 2019}\\
\end{itemize}




\section*{Awards and Honors}
\begin{itemize}
\item
	Dean's Award
	\hfill{\small Mar 2025}\\
	M.S. in Graduate School of Mathematical Science at the University of Tokyo
\item
	Japanese Government (MEXT) Scholarship
	\hfill{\small Apr 2023 -- Now}\\
	Embassy recommendation, Research student
\item
	Gold Prize in 38th Mathematics Competition for University Students in Korea
	\hfill{\small Nov 2019}\\
	1st group for math majors\\
	Sponsored by Korean Mathematics Society
\end{itemize}




\section*{Talks}
\begin{itemize}
\item
	POSTECH -- KAIST -- UNIST Undergraduate Math Club Joint Seminar
	\hfill{\small Aug 2019}\\
	Title: \emph{Diachrony of Spectra}\\
	\textcolor{gray}{Focusing on the word ``spectrum'', followed the history of how mathematical languages are created in the interdisciplinary study of physics, analysis, and geometry.}
%	Abstract:
%	\textcolor{lightgray}{In algebra, the spectrum of a ring is defined as the set of prime ideals. Focusing on the word ``spectrum'', this talk follows the history of how mathematical languages are created in the interdisciplinary study of physics, analysis, and geometry.}
\item
	POSTECH Undergraduate Mathematics Seminar
	\hfill{\small Nov 2018}\\
	Title: \emph{Dispersion for the Schr\"odinger Equation}\\
	\textcolor{gray}{Proved the dispersive inequality for the Schr\"odinger equation, using the theory of Fourier transforms and oscillatory integrals.}
%	Abstract:
%	\textcolor{lightgray}{According to theory of quantum mechanics, free particles spread out in space as time flows. Since the energy is conserved, the amplitude of wave functions consequently decay. This is due to the fact that the Schr\"odinger equation is a dispersive equation. This talk will prove it, using elementary Fourier transforms and oscillatory integrals.}
\end{itemize}




\section*{Work Experience}
\begin{itemize}
\item
	Military Service
	\hfill{\small Jan 2020 -- July 2021}\\
	1st Marine Division Band, Republic of Korea Marine Corps\\
	Military Musician (Saxophone)
\item
	Summer Experience Society (Internship Program)
	\hfill{\small Jun 2019 -- Aug 2019}\\
	Persona AI Co., Ltd.\\
	R\&D on Natural Language Processing (Chatbot development)\\
	\textcolor{gray}{Implemented probabilistic graphical models, attention model, and sentiment analyzer.}
\end{itemize}





\section*{Other Skills and Qualifications}
\begin{itemize}
\item Language: English (fluent), Japanese (fluent), Korean (native)
\end{itemize}
	

\end{document}

