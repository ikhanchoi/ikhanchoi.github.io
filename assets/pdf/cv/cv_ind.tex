\documentclass[11pt,a4paper]{article}
\usepackage[margin=1cm]{geometry}
\usepackage{fontawesome,hyperref}
\usepackage[T1]{fontenc}
\usepackage[bitstream-charter]{mathdesign}
\usepackage{luatexja}
\usepackage{enumitem}
\setlist[itemize]{label=,itemindent=-2em,leftmargin=4em}

\usepackage{titlesec}
\titleformat{\section}{\large\scshape}{\thesection}{}{}[{\titlerule[0.8pt]}]
\titlespacing{\section}{0em}{1em}{1em}
\titleformat{\subsection}{\scshape}{\thesubsection}{}{}[]
\titlespacing{\subsection}{0em}{1em}{0em}
\setlength\parindent{0pt}
\pagenumbering{gobble}

\title{\vspace{-40pt}
	Ikhan Choi\\[5pt]\small
	\faEnvelopeSquare\quad dlrgks623[at]gmail.com
	\quad$\cdot$\quad
	\faHome\quad\href{https://ikhanchoi.github.io}{ikhanchoi.github.io}
	\quad$\cdot$\quad
	\faGithub\quad\href{https://github.com/ikhanchoi}{github.com/ikhanchoi}
	\vspace{-5em}}
\date{}


\begin{document}
\maketitle



\section*{Projects}

\subsection*{Game engine}
\faGithub\quad\href{https://github.com/ikhanchoi/game-engine}{github.com/ikhanchoi/game-engine}\\[4pt]
\faArrowRight\quad
レンダリングとシミュレーションで遊べる実験室をパソコンに作りたいという目標を目指して、趣味として開発しているECS基盤自作ゲームエンジンです。
OpenGLとtinyGLTFを使います。
現在イベントシステム、カスタムアロケーター、レンダラーなどが実装されています。

\subsection*{Rendering}
\faGithub\quad\href{https://github.com/ikhanchoi/blackhole}{github.com/ikhanchoi/blackhole}\\[4pt]
\faArrowRight\quad
インターステラーのブラックホールを描画するために頑張っています。
モノマネではなく、現論文に従ってアインシュタイン方程式をフラグメントシェーダーで解こうとしています。
マルチスレッドでgifファイルを録画する機能も作りました。

\subsection*{Physics simulations}
\faGithub\quad\href{https://github.com/ikhanchoi/pba-ikhanchoi}{github.com/ikhanchoi/pba-ikhanchoi}\\[4pt]
\faArrowRight\quad
梅谷信行先生の2025年Sセメスターの授業「物理基盤アニメーション」の課題レポジトリーです。
RustとUnityで簡単なシミュレーションを行う課題で、成績は優でした。




\section*{Skills and Qualifications}
\begin{itemize}
\item Natural Language: Korean (native), Japanese (fluent), English (proficient)
\item Programming Language: C++, Python, LaTeX
\end{itemize}
	



\section*{Education}
\begin{itemize}
\item
	\textbf{The University of Tokyo}
	\hfill{\small Apr 2025 -- Now}\\
	Ph.D. program in Mathematical Science
\item
	\textbf{The University of Tokyo}
	\hfill{\small Apr 2023 -- Mar 2025}\\
	M.S. in Mathematical Science
\item
	\textbf{Pohang University of Science and Technology}
	\hfill{\small Mar 2016 -- Aug 2022}\\
	B.S. in Mathematics\\
	(Early graduation: leave of absence for three years including military service)
\item
	\textbf{Seoul Science High School}
	\hfill{\small Mar 2013 -- Feb 2016}\\
	(Specialized high school for gifted students)
\end{itemize}





\section*{Working Experiences}


\begin{itemize}
\item
	R\&D on Natural Language Processing for Chatbot development\hfill{\small Jun 2019 -- Aug 2019}\\
	Internship Program, Persona AI Co., Ltd.\\[4pt]
	\faArrowRight\quad
	Implemented variable length HMM, Bahdanau attention model, and sentiment analyzer with seq2seq.
\end{itemize}


\vfill\hfill(Continued to the next page)

\clearpage

\section*{Academic Experiences}


\subsection*{Interests}
\hspace{2em}
Functional Analysis and Homotopy Theory

\subsection*{Publications}
\begin{itemize}
\item
	\textbf{I. Choi},
	\emph{A solution to Haagerup's problem and positionve Hahn-Banach separation theorems in operator algebras},\\
	to appear in Journal of Functional Analysis, \href{https://arxiv.org/abs/2501.16832}{arXiv: 2501.16832.} (Master's thesis)\\[4pt]
	\faArrowRight\quad
	Solved a 50-year-old open problem posed by U. Haagerup, who had finished the classification of amenable type III von Neumann algebras and was one of founders of Tomita-Takesaki theory.
\item
	\textbf{I. Choi},
	\emph{Curved folding and planar cutting of simple closed curve on a conical origami},\\
	\href{https://projecteuclid.org/journals/kodai-mathematical-journal/volume-39/issue-3/Curved-folding-and-planar-cutting-of-simple-closed-curve-on/10.2996/kmj/1478073774.full}{Kodai Math. J., 39-3 (2016) 579-595.} (High school thesis)\\[4pt]
	\faArrowRight\quad
	Suggested a problem in terms of classical differential geometry of determining whether a closed curve on a plane can be realized as the intersection of a piecewise smooth isometric immersion of the plane and another plane embedded in the three-dimensional Euclidean space.
\end{itemize}



\subsection*{Awards and Honors}
\begin{itemize}
\item
	\textbf{Research Fellowship for Young Scientists} (学振 DC2)\hfill{\small Apr 2026 -- Mar 2028}\\
	Japan Society for the Promotion of Science (JSPS)
\item
	\textbf{Dean's Award} (研究科長賞)
	\hfill{\small Mar 2025}\\
	M.S. in Graduate School of Mathematical Science at the University of Tokyo
\item
	\textbf{Japanese Government (MEXT) Scholarship} (国費留学生)
	\hfill{\small Apr 2023 -- Mar 2026}\\
	Embassy recommendation
\item
	\textbf{Gold Prize} in 38th Mathematics Competition for University Students in Korea
	\hfill{\small Nov 2019}\\
	1st group for math majors\\
	Sponsored by Korean Mathematics Society
\end{itemize}


\subsection*{Teaching}

\begin{itemize}
\item
	Teaching Assistant: Linear Algebra II
	\hfill{\small Fall 2025}
\item
	Teaching Assistant: Geometry II
	\hfill{\small Fall 2024}
\item
	Teaching Assistant: Complex Analysis I
	\hfill{\small Fall 2023}
\item
	Student Mentoring: Introduction to Differential Geometry
	\hfill{\small Fall 2019}\\
\end{itemize}


\section*{Other Experiences}
\begin{itemize}
\item
	Military Service
	\hfill{\small Jan 2020 -- Jul 2021}\\
	1st Marine Division Band, Republic of Korea Marine Corps\\
	Military Musician (Saxophone)
\end{itemize}



\end{document}

